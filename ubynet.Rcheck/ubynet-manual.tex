\nonstopmode{}
\documentclass[a4paper]{book}
\usepackage[times,inconsolata,hyper]{Rd}
\usepackage{makeidx}
\makeatletter\@ifl@t@r\fmtversion{2018/04/01}{}{\usepackage[utf8]{inputenc}}\makeatother
% \usepackage{graphicx} % @USE GRAPHICX@
\makeindex{}
\begin{document}
\chapter*{}
\begin{center}
{\textbf{\huge Package `ubynet'}}
\par\bigskip{\large \today}
\end{center}
\ifthenelse{\boolean{Rd@use@hyper}}{\hypersetup{pdftitle = {ubynet: Analysis of Molecular Data}}}{}
\begin{description}
\raggedright{}
\item[Type]\AsIs{Package}
\item[Title]\AsIs{Analysis of Molecular Data}
\item[Version]\AsIs{0.2.1}
\item[Author]\AsIs{Jianwei }\email{jianwei363@gmail.com}\AsIs{}
\item[Maintainer]\AsIs{Jianwei }\email{jianwei363@gmail.com}\AsIs{}
\item[Description]\AsIs{Functions for analyzing molecular data, including reaction matching, 
molecular transformation network construction based on paired mass differences (PMD),
and biochemical transformation detection (wx:uby076).}
\item[Depends]\AsIs{R (>= 3.5.0)}
\item[Imports]\AsIs{data.table, dplyr, utils, readr, igraph, tools, ape}
\item[Suggests]\AsIs{progress, testthat}
\item[License]\AsIs{What license it uses}
\item[Encoding]\AsIs{UTF-8}
\item[RoxygenType]\AsIs{list(markdown = TRUE)}
\item[RoxygenNote]\AsIs{7.3.3}
\item[NeedsCompilation]\AsIs{no}
\end{description}
\Rdcontents{Contents}
\HeaderA{analyze\_existing\_transformations}{Analyze existing transformation files}{analyze.Rul.existing.Rul.transformations}
%
\begin{Description}
Analyze existing transformation files
\end{Description}
%
\begin{Usage}
\begin{verbatim}
analyze_existing_transformations(
  sample_name = "Dataset_Name",
  clustering_method = "average",
  output_dir = "."
)
\end{verbatim}
\end{Usage}
%
\begin{Arguments}
\begin{ldescription}
\item[\code{sample\_name}] Sample name for output files

\item[\code{clustering\_method}] Clustering method (default: "average")

\item[\code{output\_dir}] Output directory (default: current directory)
\end{ldescription}
\end{Arguments}
%
\begin{Value}
List containing tree analysis results
\end{Value}
\HeaderA{build\_mass\_pmd\_network}{Build Mass-PMD Transformation Network}{build.Rul.mass.Rul.pmd.Rul.network}
%
\begin{Description}
Constructs a network of potential biochemical transformations based on 
paired mass differences (PMD) between molecules in mass spectrometry data.
This is the original implementation.
\end{Description}
%
\begin{Usage}
\begin{verbatim}
build_mass_pmd_network(
  mol_file,
  trans_file,
  error_term = 1e-05,
  output_dir = "mass_pmd_network"
)
\end{verbatim}
\end{Usage}
%
\begin{Arguments}
\begin{ldescription}
\item[\code{mol\_file}] Character. Path to CSV file containing molecular information with 'Mass' column.

\item[\code{trans\_file}] Character. Path to CSV file containing transformation database with 'Name' and 'Mass' columns.

\item[\code{error\_term}] Numeric. Mass tolerance for matching transformations (default: 0.00001).

\item[\code{output\_dir}] Character. Directory to save results (default: "mass\_pmd\_network").
\end{ldescription}
\end{Arguments}
%
\begin{Details}
This function reads molecular mass data and a transformation database, then identifies
potential biochemical transformations by matching mass differences. The results include:
\begin{itemize}

\item{} edges.csv - Network edges showing transformations between molecules
\item{} network\_stats.txt - Network topology statistics
\item{} node\_attributes.csv - Degree information for each mass
\item{} transformation\_summary.csv - Count of each transformation type
\item{} sample\_summary.csv - Overall dataset summary

\end{itemize}

\end{Details}
%
\begin{Value}
Invisible NULL. Results are saved to files in the output directory.
\end{Value}
%
\begin{Examples}
\begin{ExampleCode}
## Not run: 
build_mass_pmd_network(
    mol_file = "MS_MolInfor1.csv",
    trans_file = "Transformation_Database_07-2020.csv",
    error_term = 0.00001,
    output_dir = "MS_MolInfor2"
)

## End(Not run)

\end{ExampleCode}
\end{Examples}
\HeaderA{build\_phylogenetic\_tree\_from\_files}{Build phylogenetic tree from transformation files}{build.Rul.phylogenetic.Rul.tree.Rul.from.Rul.files}
%
\begin{Description}
Build phylogenetic tree from transformation files
\end{Description}
%
\begin{Usage}
\begin{verbatim}
build_phylogenetic_tree_from_files(
  peak2peak_file,
  numtrans_file,
  sample_name,
  output_dir = ".",
  clustering_method = "average"
)
\end{verbatim}
\end{Usage}
%
\begin{Arguments}
\begin{ldescription}
\item[\code{peak2peak\_file}] Path to peak.2.peak CSV file

\item[\code{numtrans\_file}] Path to num.peak.trans CSV file

\item[\code{sample\_name}] Sample name for output

\item[\code{output\_dir}] Output directory

\item[\code{clustering\_method}] Clustering method for tree construction
\end{ldescription}
\end{Arguments}
%
\begin{Value}
List containing tree and network analysis results
\end{Value}
\HeaderA{classify\_Mass}{Classify Mass into product, resistant, and resistant}{classify.Rul.Mass}
%
\begin{Description}
This function takes two CSV files or data frames representing "before" and "after" datasets,
then classifies Mass into "product", "resistant", or "resistant".
\end{Description}
%
\begin{Usage}
\begin{verbatim}
classify_Mass(before, after, output_file = NULL)
\end{verbatim}
\end{Usage}
%
\begin{Arguments}
\begin{ldescription}
\item[\code{before}] Data frame or path to CSV file containing the "before" dataset.

\item[\code{after}] Data frame or path to CSV file containing the "after" dataset.

\item[\code{output\_file}] Path to save the classified results (default: NULL, no file saved).
\end{ldescription}
\end{Arguments}
%
\begin{Value}
A data frame with Mass classification.
\end{Value}
%
\begin{Examples}
\begin{ExampleCode}
# Using data frames
before_data <- data.frame(Mass = c(180.16, 46.07))
after_data <- data.frame(Mass = c(46.07, 16.04))
classify_Mass(before_data, after_data)

# Using CSV files
classify_Mass("before.csv", "after.csv", "classified_results.csv")
\end{ExampleCode}
\end{Examples}
\HeaderA{classify\_Mass\_intensity}{Classify Mass into precursors, products, and resistants}{classify.Rul.Mass.Rul.intensity}
%
\begin{Description}
This function classifies Mass based on intensity changes between "before" and "after" samples.
\end{Description}
%
\begin{Usage}
\begin{verbatim}
classify_Mass_intensity(before, after, output_file = NULL)
\end{verbatim}
\end{Usage}
%
\begin{Arguments}
\begin{ldescription}
\item[\code{before}] Data frame or path to CSV file containing the "before" dataset.

\item[\code{after}] Data frame or path to CSV file containing the "after" dataset.

\item[\code{output\_file}] Path to save the classified results (default: NULL, no file saved).
\end{ldescription}
\end{Arguments}
%
\begin{Value}
A data frame with Mass classification.
\end{Value}
\HeaderA{classify\_MolForm}{Classify MolForm into product, resistant, and disappearance}{classify.Rul.MolForm}
%
\begin{Description}
This function takes two CSV files or data frames representing "before" and "after" datasets,
then classifies MolForm into "product", "resistant", or "disappearance".
\end{Description}
%
\begin{Usage}
\begin{verbatim}
classify_MolForm(before, after, output_file = NULL)
\end{verbatim}
\end{Usage}
%
\begin{Arguments}
\begin{ldescription}
\item[\code{before}] Data frame or path to CSV file containing the "before" dataset.

\item[\code{after}] Data frame or path to CSV file containing the "after" dataset.

\item[\code{output\_file}] Path to save the classified results (default: NULL, no file saved).
\end{ldescription}
\end{Arguments}
%
\begin{Value}
A data frame with MolForm classification.
\end{Value}
%
\begin{Examples}
\begin{ExampleCode}
# Using data frames
before_data <- data.frame(MolForm = c("C6H12O6", "C2H5OH"))
after_data <- data.frame(MolForm = c("C2H5OH", "CH4"))
classify_MolForm(before_data, after_data)

# Using CSV files
classify_MolForm("before.csv", "after.csv", "classified_results.csv")
\end{ExampleCode}
\end{Examples}
\HeaderA{classify\_MolForm\_intensity}{Classify MolForm into precursors, products, and resistants}{classify.Rul.MolForm.Rul.intensity}
%
\begin{Description}
This function classifies MolForm based on intensity changes between "before" and "after" samples.
\end{Description}
%
\begin{Usage}
\begin{verbatim}
classify_MolForm_intensity(before, after, output_file = NULL)
\end{verbatim}
\end{Usage}
%
\begin{Arguments}
\begin{ldescription}
\item[\code{before}] Data frame or path to CSV file containing the "before" dataset.

\item[\code{after}] Data frame or path to CSV file containing the "after" dataset.

\item[\code{output\_file}] Path to save the classified results (default: NULL, no file saved).
\end{ldescription}
\end{Arguments}
%
\begin{Value}
A data frame with MolForm classification.
\end{Value}
\HeaderA{compare\_mass}{Compare Mass columns between two CSV files}{compare.Rul.mass}
%
\begin{Description}
This function compares the Mass columns between two CSV files with a specified
tolerance value to find matching masses.
\end{Description}
%
\begin{Usage}
\begin{verbatim}
compare_mass(
  file1,
  file2,
  output_prefix = NULL,
  output_dir = NULL,
  mass_tolerance = 0.01
)
\end{verbatim}
\end{Usage}
%
\begin{Arguments}
\begin{ldescription}
\item[\code{file1}] Character. Path to the first CSV file.

\item[\code{file2}] Character. Path to the second CSV file.

\item[\code{output\_prefix}] Character. Prefix for output files. If NULL, uses the combined file names.

\item[\code{output\_dir}] Character. Directory to save output files. If NULL, uses current directory.

\item[\code{mass\_tolerance}] Numeric. The tolerance for matching masses. Default is 0.01.
\end{ldescription}
\end{Arguments}
%
\begin{Value}
A list containing the comparison results with counts and sets of Mass values.
\end{Value}
%
\begin{Examples}
\begin{ExampleCode}
## Not run: 
result <- compare_mass("MS_MolInfor1.csv", "MS_MolInfor2.csv", 
                        output_dir = "results", mass_tolerance = 0.01)

## End(Not run)
\end{ExampleCode}
\end{Examples}
\HeaderA{compare\_molforms}{Compare MolForm columns between two CSV files}{compare.Rul.molforms}
%
\begin{Description}
This function compares the MolForm columns between two CSV files and identifies
the intersection and differences.
\end{Description}
%
\begin{Usage}
\begin{verbatim}
compare_molforms(file1, file2, output_prefix = NULL, output_dir = NULL)
\end{verbatim}
\end{Usage}
%
\begin{Arguments}
\begin{ldescription}
\item[\code{file1}] Character. Path to the first CSV file.

\item[\code{file2}] Character. Path to the second CSV file.

\item[\code{output\_prefix}] Character. Prefix for output files. If NULL, uses the combined file names.

\item[\code{output\_dir}] Character. Directory to save output files. If NULL, uses current directory.
\end{ldescription}
\end{Arguments}
%
\begin{Value}
A list containing the comparison results with counts and sets of MolForm values.
\end{Value}
%
\begin{Examples}
\begin{ExampleCode}
## Not run: 
result <- compare_molforms("MS_MolInfor1.csv", "MS_MolInfor2.csv", output_dir = "results")

## End(Not run)
\end{ExampleCode}
\end{Examples}
\HeaderA{compare\_multiple\_datasets}{Compare multiple CSV datasets}{compare.Rul.multiple.Rul.datasets}
%
\begin{Description}
This function compares multiple CSV files either by MolForm or Mass columns.
\end{Description}
%
\begin{Usage}
\begin{verbatim}
compare_multiple_datasets(
  file_list,
  output_dir = NULL,
  comparison_type = "molform",
  mass_tolerance = 0.01
)
\end{verbatim}
\end{Usage}
%
\begin{Arguments}
\begin{ldescription}
\item[\code{file\_list}] Character vector. Paths to the CSV files to compare.

\item[\code{output\_dir}] Character. Directory to save output files. If NULL, uses current directory.

\item[\code{comparison\_type}] Character. Type of comparison, either "molform" or "mass". Default is "molform".

\item[\code{mass\_tolerance}] Numeric. The tolerance for matching masses when comparison\_type is "mass". Default is 0.01.
\end{ldescription}
\end{Arguments}
%
\begin{Value}
A list containing the comparison results for all file pairs.
\end{Value}
%
\begin{Examples}
\begin{ExampleCode}
## Not run: 
files <- c("MS_MolInfor1.csv", "MS_MolInfor2.csv", "MS_MolInfor3.csv")
results <- compare_multiple_datasets(files, output_dir = "results", comparison_type = "molform")

## End(Not run)
\end{ExampleCode}
\end{Examples}
\HeaderA{complete\_transformation\_analysis}{Complete biochemical transformation analysis pipeline}{complete.Rul.transformation.Rul.analysis}
%
\begin{Description}
Complete biochemical transformation analysis pipeline
\end{Description}
%
\begin{Usage}
\begin{verbatim}
complete_transformation_analysis(
  data,
  mol,
  trans_db,
  error_term = 1e-05,
  output_dir = ".",
  sample_name = "Dataset",
  clustering_method = "average",
  build_tree = TRUE
)
\end{verbatim}
\end{Usage}
%
\begin{Arguments}
\begin{ldescription}
\item[\code{data}] Data matrix with peaks as rows and samples as columns

\item[\code{mol}] Molecular information matrix with same row names as data

\item[\code{trans\_db}] Transformation database with 'Name' and 'Mass' columns

\item[\code{error\_term}] Mass error tolerance (default: 0.000010)

\item[\code{output\_dir}] Output directory path (default: current working directory)

\item[\code{sample\_name}] Dataset name for output files (default: "Dataset")

\item[\code{clustering\_method}] Method for hierarchical clustering (default: "average")

\item[\code{build\_tree}] Whether to build phylogenetic tree (default: TRUE)
\end{ldescription}
\end{Arguments}
%
\begin{Value}
List containing transformation results, network, and phylogenetic tree
\end{Value}
\HeaderA{detect\_transformations}{Detect putative biochemical transformations across dataset}{detect.Rul.transformations}
%
\begin{Description}
Detect putative biochemical transformations across dataset
\end{Description}
%
\begin{Usage}
\begin{verbatim}
detect_transformations(
  data,
  mol,
  trans_db,
  error_term = 1e-05,
  output_dir = ".",
  sample_name = "Dataset"
)
\end{verbatim}
\end{Usage}
%
\begin{Arguments}
\begin{ldescription}
\item[\code{data}] Data matrix with peaks as rows and samples as columns

\item[\code{mol}] Molecular information matrix with same row names as data

\item[\code{trans\_db}] Transformation database with 'Name' and 'Mass' columns

\item[\code{error\_term}] Mass error tolerance (default: 0.000010)

\item[\code{output\_dir}] Output directory path (default: current working directory)

\item[\code{sample\_name}] Dataset name for output files (default: "Dataset")
\end{ldescription}
\end{Arguments}
%
\begin{Value}
List containing peak-to-peak transformations and peak profiles
\end{Value}
\HeaderA{match\_reactions\_by\_intensity}{Match reactions between two molecule datasets based on intensity filtering}{match.Rul.reactions.Rul.by.Rul.intensity}
%
\begin{Description}
This function filters precursor and product molecules based on intensity changes,
and matches them with a list of possible reaction deltas.
\end{Description}
%
\begin{Usage}
\begin{verbatim}
match_reactions_by_intensity(file1, file2, reaction_delta_file, out_dir = ".")
\end{verbatim}
\end{Usage}
%
\begin{Arguments}
\begin{ldescription}
\item[\code{file1}] Path to the first molecular information CSV (e.g., inflow).

\item[\code{file2}] Path to the second molecular information CSV (e.g., outflow).

\item[\code{reaction\_delta\_file}] Path to the reaction delta CSV file.

\item[\code{out\_dir}] Directory to save output files.
\end{ldescription}
\end{Arguments}
%
\begin{Value}
Two CSV files saved in `out\_dir`: `network\_edge.csv` and `reaction\_summary.csv`.
\end{Value}
\HeaderA{match\_reactions\_by\_mass\_difference}{Match reactions between two molecule datasets based on mass difference}{match.Rul.reactions.Rul.by.Rul.mass.Rul.difference}
%
\begin{Description}
This function filters precursor and product molecules based on intensity changes,
and matches them with a list of possible reaction mass differences.
\end{Description}
%
\begin{Usage}
\begin{verbatim}
match_reactions_by_mass_difference(
  file1,
  file2,
  reaction_delta_file,
  out_dir = ".",
  mass_tolerance = 0.005
)
\end{verbatim}
\end{Usage}
%
\begin{Arguments}
\begin{ldescription}
\item[\code{file1}] Path to the first molecular information CSV (e.g., inflow).

\item[\code{file2}] Path to the second molecular information CSV (e.g., outflow).

\item[\code{reaction\_delta\_file}] Path to the reaction delta CSV file (with mass differences).

\item[\code{out\_dir}] Directory to save output files.

\item[\code{mass\_tolerance}] Mass tolerance for matching (default: 0.005 Da).
\end{ldescription}
\end{Arguments}
%
\begin{Value}
Two CSV files saved in `out\_dir`: `network\_edge.csv` and `reaction\_summary.csv`.
\end{Value}
\HeaderA{merge\_mass\_intensity}{Merge Mass and Intensity from Multiple CSV Files (Simplified)}{merge.Rul.mass.Rul.intensity}
%
\begin{Description}
Merge Mass and Intensity from Multiple CSV Files (Simplified)
\end{Description}
%
\begin{Usage}
\begin{verbatim}
merge_mass_intensity(
  dir_path,
  output_mass_intensity,
  output_mass_elements,
  output_meta_file = NULL
)
\end{verbatim}
\end{Usage}
%
\begin{Arguments}
\begin{ldescription}
\item[\code{dir\_path}] The directory containing CSV files.

\item[\code{output\_mass\_intensity}] Path to save the merged Mass-Intensity file.

\item[\code{output\_mass\_elements}] Path to save the Mass-Elements file.

\item[\code{output\_meta\_file}] Path to save the auto-generated metadata file (optional).
\end{ldescription}
\end{Arguments}
%
\begin{Value}
List containing merged data frames.
\end{Value}
\HeaderA{merge\_molform\_intensity}{Merge Data Based on Molecular Formula (MolForm) with Filtering (Simplified)}{merge.Rul.molform.Rul.intensity}
%
\begin{Description}
Merge Data Based on Molecular Formula (MolForm) with Filtering (Simplified)
\end{Description}
%
\begin{Usage}
\begin{verbatim}
merge_molform_intensity(
  dir_path,
  output_molform_intensity,
  output_molform_elements,
  output_filtered_samples_dir,
  output_meta_file = NULL
)
\end{verbatim}
\end{Usage}
%
\begin{Arguments}
\begin{ldescription}
\item[\code{dir\_path}] The directory containing CSV files.

\item[\code{output\_molform\_intensity}] Path to save the merged MolForm-Intensity file.

\item[\code{output\_molform\_elements}] Path to save the MolForm-Elements file.

\item[\code{output\_filtered\_samples\_dir}] Directory to save individual filtered sample files.

\item[\code{output\_meta\_file}] Path to save the auto-generated metadata file (optional).
\end{ldescription}
\end{Arguments}
%
\begin{Value}
List containing merged data frames.
\end{Value}
\printindex{}
\end{document}
